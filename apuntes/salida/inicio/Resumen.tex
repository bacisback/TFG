Medir la dependencia estadística entre variables aleatorias es un problema fundamental en el área de la estadística.
Los test clasicos de dependencia como el $\rho$ de Pearson o el $\tau$ de Kendall son comúnmente aplicados debido a que son computacionalmente eficientes y están bien entendidos y estudiados, pero estos tests solamente consideran una conjunto limitado de patrones de asociación, como lineal o funciones monótonas crecientes. El desarrollo de medidas de dependencia no lineales es complejo debido a la cantidad de posibles patrones de asociación que se pueden presentar.

En este trabajo se van a presentar tres planteamientos para medir las dependencias no lineales: mediante el uso de medidas de independencia basados en kernels (HSIC), correlación canónica entre proyecciones aleatorias no lineales (RDC) y un test basado en las funciones características (DCOV)

La estructura que seguirá el proyecto es la siguiente:

Al principio de este trabajo se presenta un test de homogeneidad ,MMD, basado en empotramiento de media de las variables originales mediante transformaciones no lineales a espacios de Hilbert con un núcleo reproductivo ,RKHS, estos nuevos conocimientos se usarán para llegar al primer test de independencia ,HSIC.

En segundo lugar estudiaremos el concepto de distancia de energía e introduciremos un segundo test de independencia, DCOV. Posteriormente se pasará a estudiar la equivalencia entre este test y MMD.

Finalmente presentaremos el último test de independecia, RDC, concluyendo en la comparación de estos tres test entre ellos y otros tests existentes.




