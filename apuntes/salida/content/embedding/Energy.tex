
In this section we will define energy distance and we will use it to define a homogeneity test.
This knowledge will be used in order to formulate another independence test based on energy distance, distance covariance and distance correlation. This test is one of the most popular nowadays because of its power and the fact that it does not depend on any parameter.

\subsection{Definitions}

\begin{prop}
Let $\mathcal{F}$ and $\mathcal{G}$ be two CDFs of the independent random variables X,Y respectively and  X',Y' two iid copies of them, then:
\vspace{5mm}
$$
2\int_{-\infty}^{\infty}(\mathcal{F}(x) - \mathcal{G}(x))^{2} dx = 2\mathbb{E}\abs{X-Y} - \mathbb{E}\abs{X-X'} -\mathbb{E}\abs{Y-Y'}  
$$
\end{prop}
\begin{proof}
We will start analysing the expectations of the right hand side. We will use that for any positive random variable $ Z>0$,
$\mathbb{E}Z = \int_{0}^{\infty} \mathbb{P}(Z>z)dz$
\begin{equation}{}
\begin{split}
\mathbb{E}\abs{X-Y}
& = \int_{0}^{\infty} \mathbb{P}(\abs{X-Y} > u) du \\
&= \int_{0}^{\infty} \mathbb{P}(X - Y > u) du + \int_{0}^{\infty} \mathbb{P}(X - Y < u) du \\
&= \int_{0}^{\infty}\int_{-\infty}^{\infty} \mathbb{P}(X - Y > u|Y = y)d\mathcal{G}(y)du + \int_{0}^{\infty}\int_{-\infty}^{\infty} \mathbb{P}(X - Y < u|X = x)d\mathcal{F}(x)(y)du \\
&=\footnote{due to both having finite expectation we can apply fubini}  
\int_{-\infty}^{\infty}\int_{0}^{\infty} \mathbb{P}(X - Y > u|Y = y)du\mathcal{G}(y) + \int_{-\infty}^{\infty}\int_{0}^{\infty} \mathbb{P}(X - Y < u|X = x)du\mathcal{F}(x) \\
&= \int_{-\infty}^{\infty}\int_{0}^{\infty} \mathbb{P}(X > u + y)du\mathcal{G}(y) + \int_{-\infty}^{\infty}\int_{0}^{\infty} \mathbb{P}(Y > u + x)du\mathcal{F}(x)\\
\end{split}
\end{equation}

Now we use the change of variables z = u + y for the first integral, and w = u + x for the second one. Applying Fubini again:

\begin{equation}{}
\begin{split}
\mathbb{E}\abs{X-Y}
&= \int_{-\infty}^{\infty}\int_{y}^{\infty} \mathbb{P}(X > z)dz\mathcal{G}(y) + \int_{-\infty}^{\infty}\int_{x}^{\infty} \mathbb{P}(Y > w)dw\mathcal{F}(x)\\
&= \int_{-\infty}^{\infty} \mathbb{P}(X > z)dz\int_{y}^{\infty}\mathcal{G}(y) + \int_{-\infty}^{\infty}\mathbb{P}(Y > w)dw\int_{x}^{\infty} \mathcal{F}(x)\\
&= \int_{-\infty}^{\infty} \mathbb{P}(X > z)\mathbb{P}(Y < z)dz + \int_{-\infty}^{\infty}\mathbb{P}(Y > w)\mathbb{P}(X < w)dw\\
&= \int_{-\infty}^{\infty} [(1 - \mathcal{F}(z))\mathcal{G}(z) + (1 - \mathcal{G}(z))\mathcal{F}(z)]dz\\
&= -2\int_{-\infty}^{\infty}\mathcal{F}(z)\mathcal{G}(z)dz + \mathbb{E}\abs{X} + \mathbb{E}\abs{Y}\\
\end{split}
\end{equation}
Taking $\mathcal{F} = \mathcal{G}$ in the previous development:

$$
\mathbb{E}\abs{X-X'} =  -2\int_{-\infty}^{\infty}\mathcal{F}^{2}(z)dz + 2\mathbb{E}\abs{X}
$$

Equivalently for Y. Combining these partial results concludes the proof.

\end{proof}

\begin{defn}
Let X and Y be random variables in $\mathbb{R}^{d}$ of $\mathbb{E}\norm{X}_{d} + \mathbb{E}\norm{Y}_{d} < \infty$ the energy distance between X and Y is defined as:
\vspace{5mm}
$$\varepsilon (X,Y) = 2\mathbb{E}\norm{X-Y}_{d} - \mathbb{E}\norm{X-X'}_{d} - \mathbb{E}\norm{Y-Y'}_{d}$$
where X' and Y' are i.i.d copies of X and Y respectively.
The energy distance can also be defined in terms of the characteristic functions. In fact, it can be seen as a weighted $\mathcal{L}_{2}$ distance between characteristic functions.
\end{defn}
\begin{prop}
Given two independent d-dimensional random variables X and Y, with distributions $\mathbb{P}$ and $\mathbb{Q}$ respectively such that $\mathbb{E}\norm{X}_{d} + \mathbb{E}\norm{Y}_{d} < \infty$ he energy distance between X and Y can be written as:
\vspace{5mm}
$$\varepsilon(X,Y) = \frac{1}{c_{d}}\int_{\mathbb{R}^{d}}\frac{\abs{\phi_{\mathbb{P}}(t)-\phi_{\mathbb{Q}}(t)}^{2}}{\norm{t}^{d+1}_{d}}$$
where
$$c_{d} = \frac{\pi^{\frac{d+1}{2}}}{\Gamma(\frac{d+1}{2})}$$
being $\Gamma(\cdot)$ the gamma funcion
\end{prop}
To prove this proposition we need the following lemma.
\begin{lem}
For all$x \in \mathbb{R}^{d}$ then:
\vspace{5mm}
$$\int_{\mathbb{R}^{d}}\frac{1-cos(tx)}{\norm{t}^{d+1}_{d}} dt = c_{d}\norm{x}_{d}$$
where tx is the inner product of t and x.
\end{lem}
\begin{proof}
We will begin by applying the following transformation: $z_{1} = \frac{tx}{\norm{x}_{d}}$
followed by the following change of variables:
$s = z\norm{x}_{d}$
\begin{equation}{}
\begin{split}
\int_{\mathbb{R}^{d}}\frac{1-cos(tx)}{\norm{t}^{d+1}_{d}} dt 
&= \int_{\mathbb{R}^{d}}\frac{1-cos(z\norm{x}_{d})}{\norm{z}^{d+1}_{d}} dt \\
&= \int_{\mathbb{R}^{d}}\frac{1-cos(s)}{\frac{\norm{s}_{d}}{\norm{x}_{d}}^{d+1} \norm{x}^{d}_{d}} dt \\
&=\norm{x}_{d}\int_{\mathbb{R}^{d}}\frac{1-cos(s)}{\norm{s}^{d+1}_{d}}ds \\
&=\norm{x}_{d}\frac{\pi^{\frac{d+1}{2}}}{\Gamma(\frac{d+1}{2})}
\end{split}
\end{equation}
\end{proof}
\begin{proof}
(Proposition 1.3.2) Let $\overline{\phi_{\mathbb{P}}(t)}$ denote the complex conjugate of the characteristic function.
\begin{equation}{}
\begin{split}
\abs{\phi_{\mathbb{P}}(t)-\phi_{\mathbb{Q}}(t)}^{2} = 
& = (\phi_{\mathbb{P}}(t)-\phi_{\mathbb{Q}}(t))\overline{(\phi_{\mathbb{P}}(t)-\phi_{\mathbb{Q}}(t))} \\
& = (\phi_{\mathbb{P}}(t)-\phi_{\mathbb{Q}}(t))(\overline{\phi_{\mathbb{P}}(t)}-\overline{\phi_{\mathbb{Q}}(t)})\\
&= \phi_{\mathbb{P}}(t)\overline{\phi_{\mathbb{P}}(t)} - \phi_{\mathbb{P}}(t)\overline{\phi_{\mathbb{Q}}(t)} - \phi_{\mathbb{Q}}(t)\overline{\phi_{\mathbb{P}}(t)} + 
\phi_{\mathbb{Q}}(t)\overline{\phi_{\mathbb{Q}}(t)} \\
&=\mathbb{E}[e^{itX}e^{-itX'}]-\mathbb{E}[e^{itX}e^{-itY}]
-\mathbb{E}[e^{itY}e^{-itX}] + \mathbb{E}[e^{itY}e^{-itY'}]\\
&=\mathbb{E}[e^{it(X-X')}-e^{it(Y-X)}-e^{it(X-Y)}+e^{it(Y-Y')} \\
&=\mathbb{E}[cos(t(X-X')) + isin(t(X-X'))-cos(t(Y-X))-isin(t(Y-X)) -cos(t(X-Y)) \\
&-isin(t(X-Y)) + cos(t(Y-Y'))+isin(t(Y-Y')) \\
&\text{sin(X) = -sin(-X), cos(X) = cos(-X), sin (x- y)=sin(x)cos(y)- cos(x)sin(y)}\\
&=\mathbb{E}[cos(t(X-X')) - 2cos(t(Y-X)) + cos(t(Y-Y')) \\
& + isin(t(X-X')) + isin(t(Y-Y'))\\
&= 
\end{split}
\end{equation}
Applying Fubini and the previous lemma:
\begin{equation}{}
\begin{split}
\int_{\mathbb{R}^{d}}\frac{\abs{\phi_{\mathbb{P}}(t)-\phi_{\mathbb{Q}}(t)}^{2}}{\norm{t}^{d+1}_{d}}
&=\int_{\mathbb{R}^{d}}\frac{\mathbb{E}[2(1-cos(t(Y-X)))-(1-cos(t(X-X')))]}{\norm{t}^{d+1}_{d}} dt \\
&=2\mathbb{E}[\int_{\mathbb{R}^{d}}\frac{1-cos(t(Y-X))}{\norm{t}^{d+1}_{d}}] - \mathbb{E}[\int_{\mathbb{R}^{d}}\frac{1-cos(t(X-X'))}{\norm{t}^{d+1}_{d}} - - \mathbb{E}[\int_{\mathbb{R}^{d}}\frac{1-cos(t(Y-Y'))}{\norm{t}^{d+1}_{d}} \\
&=2\mathbb{E}[c_{d}\norm{Y-X}] - \mathbb{E}[c_{d}\norm{X-X'}] -\mathbb{E}[c_{d}\norm{Y-Y'}] \\
&= c_{d}(2\mathbb{E}[\norm{Y-X}] - \mathbb{E}[\norm{X-X'}] -\mathbb{E}[\norm{Y-Y'}])\\
&= c_{d}\varepsilon(X,Y)
\end{split}
\end{equation}
\end{proof}
It is easy to see that the energy distance only vanishes when the distributions are equal, since it is equivalent to having equal characteristic functions.
\subsection{Application to an independence test}
In this subsection we will use the knowledge aquired above to develop a new independence test. This new test is called distance covariance (DCOV), it's name comes from the fact that it is a generalization of the classical product-moment covariance.

We will start by defininf the independence test. Given the random vectors $X\in\mathbb{R}^{dx},Y\in\mathbb{R}^{dy}$, distributions $\mathbb{P}_{X} and \mathbb{P}_{Y}$ respectively.Let $\phi_{\mathbb{P}_{X}}, \phi_{\mathbb{P}_{Y}}$ denote their characteristic functions and $\phi_{\mathbb{P}_{XY}}$ the characteristic function of the joint distribution. X and Y are independent if and only if 
$\phi_{\mathbb{P}_{X}}\phi_{\mathbb{P}_{Y}} = \phi_{\mathbb{P}_{XY}}$. The covariance energy test is based on measuring a distance between these functions.

First we need to generalize the energy distance expression for random vectors of different dimensions. As defined earlier this expression is obtained from a weighted $mathcal{L}_{2}$-distance, imposing rotation invariance and scale equivariance, the energy distance is:
\vspace{5mm}

$$\varepsilon(X,Y) = \frac{1}{c_{d_{x}}c_{d_{y}}}\int_{\mathbb{R}^{d_{x} + d_{y}}}\frac{\abs{\phi_{\mathbb{P}}(t)-\phi_{\mathbb{Q}}(t)}^{2}}{\norm{t}^{d_{x}+1}_{d_{x}}\norm{s}^{d_{y}+1}_{d_{y}}}dtds$$

Where $c_{d}$ is defined as before.
The distance covariance is defined by replacing $\phi_{\mathbb{P}}$ and $\phi_{\mathbb{Q}}$ in the previous formula with characteristic functions of the joint distribution and the product of the marginals respectively.
\begin{defn}
The distance covatiance, DCOV, between random vectors X and Y, with $\mathbb{E}\norm{X}_{d_{x}} + \mathbb{E}\norm{Y}_{d_{y}} < \infty$, is the nonnegative number $\nu^{2}(X,Y)$ defined by:
\vspace{5mm}
$$
\nu^{2}(X,Y) = \norm{\phi_{\mathbb{P}_{X,Y}}(t,s)-\phi_{\mathbb{P}_{X}}(t)\phi_{\mathbb{P}_{Y}}(s)}^{2}_{w} = \frac{1}{c_{d_{x}}c_{d_{y}}}\int_{\mathbb{R}^{d_{x} + d_{y}}}\frac{\abs{\phi_{\mathbb{P}_{X,Y}}(t,s)-\phi_{\mathbb{P}_{X}}(t)\phi_{\mathbb{P}_{Y}}(s)}^{2}}{\norm{t}^{d_{x}+1}_{d_{x}}\norm{s}^{d_{y}+1}_{d_{y}}} dtds
$$
\end{defn}
\begin{defn}
The distance correlation, DCOR, between random vectors X and Y, ith $\mathbb{E}\norm{X}_{d_{x}} + \mathbb{E}\norm{Y}_{d_{y}} < \infty$, is the nonnegative number $\mathcal{R}(X,Y)$ defined by:

$$
\mathcal{R}(X,Y) = \left \{
	\begin{array}{c} 
		\frac{\nu^{2}(X,Y)}{\sqrt{\nu^{2}(X)\nu^{2}(Y)}} \text{ if } \nu^{2}(X)\nu^{2}(Y) > 0 \\ 
		\\
		0 \text{ if } \nu^{2}(X)\nu^{2}(Y) = 0 
	\end{array}
	\right.  
$$
\end{defn}
The distance covariance, like the energy distance, can be expressed using expectations.
\begin{lem}
Let (X,Y),(X',Y'),(X'',Y'') $\sim \mathbb{P}_{XY}$ be iid copies of (X,Y), it holds that:
\begin{equation}
\begin{split}
\nu^{2}(X,Y) = &
\mathbb{E}_{XY}\mathbb{E}_{X'Y'}\norm{X-X'}_{d_{x}}\norm{Y-Y'}_{d_{y}} + \mathbb{E}_{X}\mathbb{E}_{X'}\norm{X-X'}_{d_{x}}\mathbb{E}_{Y}\mathbb{E}_{Y'}\norm{Y-Y'}_{d_{y}} \\
& -2\mathbb{E}_{XY}[\mathbb{E}_{X'}\norm{X-X'}_{d_{x}}\mathbb{E}_{Y'}\norm{Y-Y'}_{d_{y}} \\
\end{spliit}
\end{equation} \label{DCOV}
This proof is similar to the one of proposition 1.3.2. therefore we will leave it for the interested readers.
\end{lem}

\subsection{Statistics}
Given the definition obtained in \ref{DCOV} now we will define it's statistics:
\begin{theorem}

\end{theorem}