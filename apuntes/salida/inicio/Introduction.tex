How  to  measure  dependence  of  variables  is  a  classical yet fundamental problem in statistics.  Starting with the Galton’s work of Pearson’s correlation coefficient [Stigler, 1989] for measuring linear dependence,  many  techniques  have  been proposed, which are of fundamental importance in scientific fields such as physics, chemistry, biology, and economics.
In Statistics, probability measures are used in a variety of applications, such as hypothesis testing, density estimation or Markov chain monte carlo. We will focus on hypothesis testing, mainly in homogeneity testing. 
The goal in homogeneity testing is to accept or reject the null hypothesis $\mathcal{H}_{0}$:$\mathbb{P}=\mathbb{Q}$, versus the alternative hypothesis $\mathcal{H}_{1}$:$\mathbb{P}\neq\mathbb{Q}$, for a class of probability distributions $\mathbb{P}$ and $\mathbb{Q}$. For this purpose we will define a metric $\gamma$ such that testing the null hypothesis is equivalent to testing for $\gamma(\mathbb{P}\mathbb{Q}) = 0$. We are specially interested in testing for independence between random vectors, which is a particular case of homogeneity testing, using $\mathbb{P} = \mathbb{P}_{\mathcal{XY}}$ and $\mathbb{Q} = \mathbb{P}_{\mathcal{X}}\cdot\mathbb{P}_{\mathcal{Y}}$. An example of a practical application of this tests is Principal Component Analysis (PCA), which is a statistical procedure that converts a set of observations of possibly correlated variables into a set of linearly uncorrelated variables called principal components. 

In this work three main approches of non-linear dependence measures will be presented: by using kernel independence measures (HSIC), canonical correlation between random non-linear projections (RDC) and a characteristic function based test (DCOV).

The structure of the work will go as it follows:

In the begining of the work, which is composed of Chapters , an homogeneity test ,MMD, based on mean embeddings of the original variables through non-linear transformations into Hilbert spaces with reproducing kernel ,RKHS, will be introduced this new intuitions will lead us to our first independence test ,HSIC.

Secondly we will study the concept of energy distance and introduce the second independence test ,DCOV. Followed by an study of the equivalence of this tests with MMD.

Finally RDC will be presented, concluding with a comparison of this three tests between them and with other tests. 

\subsection{Reproducing Kernel Hilbert Spaces (RKHS)}



