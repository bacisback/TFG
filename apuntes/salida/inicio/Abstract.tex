Measuring statistical dependence between random variables is a fundamental problem in statistics.
Classical tests of dependence such as Pearson’s $\rho$ or Kendall’s $\tau$ are widely applied due to  being computationally efficient and theoretically well understood, however they consider only a limited class of association patterns, like linear or monotonically increasing functions. The development of non-linear dependence measures is challenging because of the radically larger amount of possible association patterns.

In this work three main approches of non-linear dependence measures will be presented: by using kernel independence measures (HSIC), canonical correlation between random non-linear projections (RDC) and a characteristic function based test (DCOV).

The structure of the work will go as it follows:

In the begining of the work, which is composed of Chapters , an homogeneity test ,MMD, based on mean embeddings of the original variables through non-linear transformations into Hilbert spaces with reproducing kernel ,RKHS, will be introduced, this new intuitions will lead us to our first independence test ,HSIC.

Secondly we will study the concept of energy distance and introduce the second independence test ,DCOV. Followed by an study of the equivalence of this tests with MMD.

Finally RDC will be presented, concluding with a comparison of this three tests between them and with other tests. 


